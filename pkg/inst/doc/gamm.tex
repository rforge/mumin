\documentclass{article}
\usepackage[utf8]{inputenc}
\addtolength{\textwidth}{1.25in}
\addtolength{\oddsidemargin}{-.75in}
\setlength{\evensidemargin}{\oddsidemargin}

%\VignettePackage{MuMIn}
%\VignetteIndexEntry{Model selection with GAMM}
%\VignetteDepends{mgcv,gamm4}

\usepackage{url}
\newcommand{\code}[1]{{\tt #1}}
\newcommand{\pkg}[1]{{\tt #1}}
\newcommand{\sQuote}[1]{{`#1'}}
\newcommand{\dQuote}[1]{{``#1''}}

\title{Model selection with \pkg{MuMIn} and GAMM}
\date{\today}
\author{Kamil Bartoń}
\usepackage{Sweave}
\begin{document}
\maketitle


\section{Extending \pkg{MuMIn}'s functionality to support \code{gamm} }

The two principal functions in \pkg{MuMIn}, \code{model.avg} and \code{dredge} rely on 
availability of methods for a several generic function for the class of
the given fitted model object. 
These generic functions include ones defined in package \code{stats} 
(\code{logLik}, \code{formula}, \code{nobs}, 
and optionally \code{deviance} which may simply return \code{NULL}), as well 
as ones defined in \pkg{MuMIn} itself (\code{coeffs}, 
\code{getAllTerms} and \code{tTable}). In some cases the default methods may 
work as well.

In case of \code{gamm} and \code{gamm4}, the returned object has no special class, 
it is a list with two items: \code{lme} or \code{mer}, and \code{gam} 
(with some information stripped from it). 
Therefore no specific methods can be applied.

The solution is to provide a wrapper function for \code{gamm} that evaluates 
the model and adds a class attribute onto it, e.g.:
\begin{Schunk}
\begin{Sinput}
> gamm <- function(...) structure(c(mgcv::gamm(...), list(call = match.call())), 
+     class = c("gamm", "list"))
\end{Sinput}
\end{Schunk}
similarly for \code{gamm4} (but assign the same class \code{gamm}):
\begin{Schunk}
\begin{Sinput}
> gamm4 <- function(...) structure(c(gamm4::gamm4(...), list(call = match.call())), 
+     class = c("gamm", "list"))
\end{Sinput}
\end{Schunk}

As they have the same names as the actual functions, so it is invisible for
the user, and masks the original functions on the level of \code{.GlobalEnv}.

In addition, the wrappers add a \code{call} element, containing the original call to the 
wrapper function. It is not necessary, but makes things easier later on for \code{dredge}.

Once we have an object of class \code{gamm}, it is possible to provide methods for it.
First let us define the generic methods from \pkg{stats}.

\begin{Schunk}
\begin{Sinput}
> logLik.gamm <- function(object, ...) logLik(object[[if (is.null(object$lme)) "mer" else "lme"]], 
+     ...)
> formula.gamm <- function(x, ...) formula(x$gam, ...)
> nobs.gamm <- function(object, ...) nobs(object$gam, ...)
\end{Sinput}
\end{Schunk}

It should be noted here that the issue of what the log-likelihood for GAMM should
be is not entirely clear. The documentation for \code{gamm} states that the log-likelihood 
of \code{lme} is not the one of the fitted GAMM. However, comparing alternative models 
shows some evidence that it
may be still appropriate for \code{gamm}. Namely the log-likelihood of fitted \code{lme}, and one of the 
\code{lme} part of \code{gamm} (including only linear terms to make the comparison adequate), 
have identical values.
\begin{Schunk}
\begin{Sinput}
> dat <- gamSim(6, n = 100, scale = 0.2, dist = "gaussian")
\end{Sinput}
\begin{Soutput}
4 term additive + random effectGu & Wahba 4 term additive model
\end{Soutput}
\begin{Sinput}
> fm1 <- gamm(y ~ x0 + x1 + x2 + x3, data = dat, random = list(fac = ~1), 
+     method = "ML")
> fm2 <- lme(y ~ x0 + x1 + x2 + x3, data = dat, random = list(fac = ~1), 
+     method = "ML")
> logLik(fm1$lme)
\end{Sinput}
\begin{Soutput}
'log Lik.' -214.5197 (df=7)
\end{Soutput}
\begin{Sinput}
> logLik(fm2)
\end{Sinput}
\begin{Soutput}
'log Lik.' -214.5197 (df=7)
\end{Soutput}
\end{Schunk}

Likewise is in the generalised case of \code{gamm4} and \code{lmer}:
\begin{Schunk}
\begin{Sinput}
> dat <- gamSim(6, n = 100, scale = 0.2, dist = "poisson")
\end{Sinput}
\begin{Soutput}
4 term additive + random effectGu & Wahba 4 term additive model
\end{Soutput}
\begin{Sinput}
> fmg1 <- gamm4(y ~ x0 + x1 + x2 + x3, family = poisson, data = dat, 
+     random = ~(1 | fac))
> fmg2 <- lmer(y ~ x0 + x1 + x2 + x3 + (1 | fac), family = poisson, 
+     data = dat)
> logLik(fmg1$mer)
\end{Sinput}
\begin{Soutput}
'log Lik.' -460.5087 (df=6)
\end{Soutput}
\begin{Sinput}
> logLik(fmg2)
\end{Sinput}
\begin{Soutput}
'log Lik.' -460.5087 (df=6)
\end{Soutput}
\end{Schunk}

Similarly, comparison of \code{gamm4} with a smooth term,
with fixed two degrees of freedom gives
log-likelihood which is very close to that of \code{lmer} that includes a linear and quadratic term. 
\begin{Schunk}
\begin{Sinput}
> fmgs1 <- gamm4(y ~ x0 + s(x1, k = 3, fx = TRUE) + x2 + x3, family = poisson, 
+     data = dat, random = ~(1 | fac))
> fmgs2 <- lmer(y ~ x0 + x1 + I(x1^2) + x2 + x3 + (1 | fac), family = poisson, 
+     data = dat)
> logLik(fmgs1$mer)
\end{Sinput}
\begin{Soutput}
'log Lik.' -459.4854 (df=7)
\end{Soutput}
\begin{Sinput}
> logLik(fmgs2)
\end{Sinput}
\begin{Soutput}
'log Lik.' -460.3622 (df=7)
\end{Soutput}
\end{Schunk}

Normally, the object returned by \code{gam} inherits also from glm, so the \code{nobs} method 
for \code{glm} is called, but in case of \code{gamm} the \code{gam} element has only class \code{gam}, so we 
need to define method directly (it just calls \code{nobs.glm}):
\begin{Schunk}
\begin{Sinput}
> nobs.gam <- function(object, ...) stats:::nobs.glm(object, ...)
\end{Sinput}
\end{Schunk}

Methods for generic functions defined in \pkg{MuMIn}:
\begin{Schunk}
\begin{Sinput}
> coeffs.gamm <- function(model) coef(model$gam)
> getAllTerms.gamm <- function(x, ...) getAllTerms(x$gam)
> tTable.gamm <- function(model, ...) tTable(model$gam)
\end{Sinput}
\end{Schunk}
(the name \code{tTable} is somewhat misleading, as the \code{data.frame} 
returned does not need to contain \emph{t}-values, two columns are obligatory:
\sQuote{Estimate} and \sQuote{Std. Error})

\section{Model selection}

Now we have all the prerequisites to proceed with the model selection:
\begin{Schunk}
\begin{Sinput}
> fmgs1 <- gamm4(y ~ s(x0) + s(x1) + s(x2) + s(x3), family = poisson, 
+     data = dat, random = ~(1 | fac))
> (dd <- dredge(fmgs1))
\end{Sinput}
\begin{Soutput}
Global model: gamm4(y ~ s(x0) + s(x1) + s(x2) + s(x3), family = poisson, data = dat, 
    random = ~(1 | fac))
---
Model selection table 
   (Int) s(x0) s(x1) s(x2) s(x3) k  AICc   delta    weight
8  3.097 +     +     +            8  197.6    0.000 0.901 
16 3.096 +     +     +     +     10  202.0    4.427 0.099 
7  3.119       +     +            6  258.5   60.930 0.000 
15 3.117       +     +     +      8  264.5   66.910 0.000 
14 3.123 +           +     +      8  526.7  329.200 0.000 
13 3.136             +     +      6  541.4  343.800 0.000 
6  3.139 +           +            6  546.5  348.900 0.000 
5  3.154             +            4  570.4  372.800 0.000 
12 3.148 +     +           +      8  698.4  500.800 0.000 
4  3.189 +     +                  6  911.5  713.900 0.000 
11 3.200       +           +      6 1007.0  809.800 0.000 
10 3.209 +                 +      6 1153.0  955.000 0.000 
3  3.237       +                  4 1210.0 1012.000 0.000 
2  3.249 +                        4 1315.0 1117.000 0.000 
9  3.261                   +      4 1410.0 1213.000 0.000 
1  3.304                          2 1624.0 1426.000 0.000 
\end{Soutput}
\begin{Sinput}
> summary(model.avg(dd, subset = delta <= 8))
\end{Sinput}
\begin{Soutput}
Call:  model.avg(object = dd, subset = delta <= 8)


Model summary:
        Deviance   AICc Delta Weight
1+2+3            197.56  0.00    0.9
1+2+3+4          201.99  4.43    0.1

Variables:
    1     2     3     4 
s(x0) s(x1) s(x2) s(x3) 

Model-averaged coefficients:
            Coefficient         SE z value Pr(>|z|)    
(Intercept)   3.097e+00  3.174e-01   9.758  < 2e-16 ***
s(x0).1      -2.940e-01  1.109e-01   2.651 0.008028 ** 
s(x0).2      -1.515e-01  3.174e-01   0.477 0.633289    
s(x0).3       9.648e-03  7.350e-02   0.131 0.895559    
s(x0).4      -1.059e-01  1.754e-01   0.604 0.545859    
s(x0).5       3.116e-02  5.853e-02   0.532 0.594481    
s(x0).6      -1.395e-01  1.584e-01   0.880 0.378622    
s(x0).7       5.167e-02  6.826e-02   0.757 0.449082    
s(x0).8       5.924e-01  3.963e-01   1.495 0.134994    
s(x0).9       2.113e-01  1.749e-01   1.208 0.227065    
s(x1).1       9.901e-03  3.950e-02   0.251 0.802062    
s(x1).2       1.714e-02  6.347e-02   0.270 0.787079    
s(x1).3       5.271e-03  2.023e-02   0.261 0.794462    
s(x1).4       1.647e-02  3.449e-02   0.477 0.633094    
s(x1).5      -7.000e-03  1.422e-02   0.492 0.622478    
s(x1).6      -1.773e-02  3.088e-02   0.574 0.565908    
s(x1).7      -2.885e-03  8.010e-03   0.360 0.718730    
s(x1).8      -9.510e-02  1.032e-01   0.922 0.356607    
s(x1).9       3.805e-01  5.020e-02   7.581  < 2e-16 ***
s(x2).1       1.089e+00  1.952e-01   5.578  < 2e-16 ***
s(x2).2      -2.615e+00  6.846e-01   3.819 0.000134 ***
s(x2).3      -1.712e+00  2.503e-01   6.843  < 2e-16 ***
s(x2).4      -4.566e-01  4.323e-01   1.056 0.290877    
s(x2).5       1.995e-01  1.751e-01   1.139 0.254537    
s(x2).6      -9.772e-01  4.896e-01   1.996 0.045955 *  
s(x2).7      -4.775e-02  2.428e-01   0.197 0.844070    
s(x2).8       3.361e+00  1.074e+00   3.128 0.001758 ** 
s(x2).9       1.162e+00  4.706e-01   2.470 0.013527 *  
s(x3).1       0.000e+00  4.560e-07   0.000 1.000000    
s(x3).2       6.407e-36  7.213e-07   0.000 1.000000    
s(x3).3       5.520e-37  1.428e-07   0.000 1.000000    
s(x3).4      -3.223e-36  4.225e-07   0.000 1.000000    
s(x3).5       5.615e-37  1.276e-07   0.000 1.000000    
s(x3).6      -2.195e-36  3.703e-07   0.000 1.000000    
s(x3).7       1.093e-36  1.875e-07   0.000 1.000000    
s(x3).8      -1.437e-20  1.266e-06   0.000 1.000000    
s(x3).9       1.391e-03  7.738e-03   0.180 0.857333    
---
Signif. codes:  0 '***' 0.001 '**' 0.01 '*' 0.05 '.' 0.1 ' ' 1 

Non-present predictors taken to be zero 

Relative variable importance:
s(x0) s(x1) s(x2) s(x3) 
  1.0   1.0   1.0   0.1 
\end{Soutput}
\end{Schunk}

\end{document}
